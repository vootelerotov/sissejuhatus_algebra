\documentclass[12pt]{book}

\usepackage{amsmath}
\usepackage{enumerate}

\begin{document}



\chapter{I loeng}

\section{Meenutusi Algebra I-st}

$A \neq \emptyset$
$A^n = A x A x ... A = \{(a_1,\cdots,a_n)| a_i \in A \}$

n - tuple
$|A^0| = 1 (A^0 = \{ \emptyset \}$
$\omega: A^n \rightarrow A$ 
n-kohaline algebraline tehe hulgal A

n-aarne:
\begin{enumerate}
\item n=1 unaarne tehe
\item n=2 binaarne tehe
\item n=0 nullarne tehe
\end{enumerate}


\section{$\Omega$-algebra}

Def. 1.1.1 Hulka $\Omega$ nimetakse t\"u\"ubiks ehk signatuuriks kui ta on esitatud mittelõikuvate alamhulkade $\Omega_1 yhend \Omega_2...$

Def 1.1.2 Olgu $\Omega tüüp$. Mittet\"uhja hulka A nimetatakse $\Omega$-algebraks, kui iga $a$ korral igale $\omega \in \Omega_n$ vastab n-aarne tehe hulgal A, mida tähistatakse sama s\"mboliga $\omega$. 

Kui tahetakse rõhutada, mis t\"u\"upi algebraga on tegemist, siis tähistatakse $\Omega$ algebrad paarina $(A;\Omega)$

Näited:

\begin{enumerate}
\item R\"uhmoid - hulk \"he binaarse tehtega, st. $\Omega= \Omega_2 = \{*\}$
\item Poolr\"uhm - sama signatuur mis r\"uhmoidil
\item Monoid - \"uhikelemendiga poolr\"uhm, vaatame seda tihti laiemal signatuuriga, $\Omega = \Omega_0 yhend \Omega_2, \Omega_0 = \{1\}$ - \"uhikelemendi fikseerimine, $\Omega_2 = \{*\}$
\item R\"uhm - saab kirjeldada eelnevate signatuuride kaudu, aga parem kirjeldada järgnevalt: $\Omega = \Omega_0 yhend \Omega_1 yhend \Omega_2$, kus $\Omega_1 = \{ ^{-1}\}$
\item Ring - algebralinie struktuur signatuuriga(signatuuris): $\Omega = \Omega_0 yhend \Omega_1 yhend \Omega_2$, kus $\Omega_2 = \{+,*\}, \Omega_1 = {-(vastandelemendi võtmine}, \Omega_0 = {0,1}$
\item Vektorruum - struktuur signatuuriga:
$\Omega = \Omega_0 yhend \Omega_1 yhend \Omega_2$, kus
$\Omega_2 = \{+\}, \Omega_1 = \{-(vastandelement, pole vajalik kui skalaariga korrutamine sissetoodud)\} yhend {\alpha * | \alpha \in K}, \Omega_0 = \{0\} - samuti avaldatav skalaariga korrutimase kaudu$ 
\end{enumerate}

\section{Morfismid}

Def 1.2.1 Kujutust $\phi$ $\Omega$-algebrast $A$ $\Omega$-algebrasse $B$ nimetatakse homomorfismideks, kui iga $n, \omega \in \Omega_n$ ja suvaliste $a_1,...,a_n \in A$ korral kehtib võrdus $\phi(\omega(a_1,...,a_n)) = \omega(\phi(a_1),\cdots, \phi(a_n))$

$Hom(A,B) - \{\phi | \phi on homoformism A-st B-sse\}$

Näide:

Olgu A, B sellised $\Omega = \{1\}, \{ ^{-1}\}, \{*\}$
$\phi : A \rightarrow B homoformism
\phi(1) = 1, \phi(1) = \phi(1*1) = \phi(1)*\phi(1) \implies \phi(1) = 1 (kolmanda põhjal)
\phi(x^{-1}) = \phi(x)^{-1}, 1 = \phi(1) = \phi(x^{-1}x) = \phi(x^{-1})\phi(x) (kolmanda põhjal)
\phi(xy) = \phi(x)\phi(y)$

Taandub kolmanda omanduse kontrollimisele.


Lineaarkujutis on vektorruumide isomorfism.

Olgu meil $\Omega$-algebrad A,B,C ning nende homoformismid $\phi : A \rightarrow B$, $\psi : B \rightarrow C, (\psi \phi ) = \psi (\phi ( x)), x \in A$. Siis see kompositsioon on samuti homoformism (kui teda saab nii defineerida). 

Tõestame : Veendume, et $(\psi\phi)(\omega(a_1,\cdots, a_n)) = \omega((\psi \phi)(a_1,\cdots,a_n))$. See on samaväärne sellega, te $\psi(\phi(\omega(a_1,\cdots,a_n))) = \psi(w(\phi(a_1),\cdots, \phi(a_n)) ...$

Endomorfism (End(A) = Hom(A,A)).

Lause 1.2.2 (End(A);*) on monoid. 

Tõestus: Assotatiivsus on selge, tuleneb homoformismide omadustest. \"Uhikelement ? $id_A : A \rightarrow B, id_{A}(x) = x, x \in A$

Def 1.2.3 Bijektiivne homomorfismi nimetatakse isomorfismiks. 

Lause 1.2.3 Isomorfism on ekvivalentsiseos kõigi $\Omega$-algebrade klassis, ehk ta on reflektsiivne, s\"ummeetriline ja transitiivne. 

Tõestus:
\begin{enumerate}
\item Refleksiivsus, st. A $isom$ A, $id_A: A \rightarrow A$
\item S\"ummeetria. Olgu $\phi: A \rightarrow B$ isomorfism. Vaja $\ksi : B \rightarrow A$ mis oleks isomorfism. Valime selleks $\phi^{-1}$ Vaja näidata, et iga  $b_1,\cdots,b_n \in B$ korral $\phi^{-1} (\omega(b_1,\cdots, b_n)) = \omega(\phi^{-1}(b_1), \cdots, \phi^{-1}(b_2))$ Rakendame mõlemale poole $\phi$. $\phi(\omega(b_1,\cdots, b_n)) = \phi \omega(\phi^{-1}(b_1), \cdots, \phi^{-1}(b_2)) .... $
\item Transitiivsus - ise! 
\end{enumerate}

Isomorfismi tähtsus. Kui meid huvitab tehe ja tema omadused, siis need jäävad samaks isomorfismi klassi täpsusega. 

$Aut(A)$ 

Lause 1.2.4 Aut(A) on rühm. 

Tõestus: 
$\phi,\psi \in AutA,

 \psi \phi \in End(A),
 
 \phi \psi \in AutA
 
 id_a \in AutA
 
 \phi in AutA \implies \phi^{-1} \in AutA
 $
 
 Näited:
 
 C kompleksarue korpus
 $\phi: C \rightarrow C, \phi(\alpha) = \overline{\alpha}$
 
 G suvaline r\"uhm
 
 $g \in G$
 
 $\phi : G \rightarrow ...$
 
 $\phi(x) = g^{-1} x g $
 
 \section{Alamalgebra}

 def 1.3.1. Mõte : $B \subset A, b_1,\cdots,b_n : \omega^B (b_1,\cdots,b_n) = \omega^A (b_1,\cdots,b_n) ( \in B).$ 
 Algebra alamhulk, mis on kinnine tehete suhtes on alamalgebra.
 
Näide: $(A;*)$ poolr\"uhm. $B \subset A, x,y \in B \implies xy \in B $ , kui $B = \emptyset$, siis ei ole alamalgebra aga rahuldab definitsiooni. Täiendame: 
  Algebra \textbf{mittet\"hi} alamhulk, mis on kinnine tehete suhtes on alamalgebra.
  
 $B \leq A \iff B on A alamalgebra$
 
$ B \leq A, \jota : B \rightarrow A

\jota(x) = x, x \in B

\jota \in Hom(B,A)$, $\jota$ \"uks\"uhene. 


1.3.1 tõestus:
 
$\phi \in Hom(B,A)$
$\phi $ \"uks\"uhene.
$\phi(B) \subset A$
$B \rightarrow \phi(B) = \{ \phi(x) | x \in B \}$

$phi \in Hom(A,B)$
$C \leq A , D \leq B $

$\phi(C) \leq B , \phi^{-1}(B) \leq A$

Esimese ise. Teine:

$\phi^{-1}(B) \leq A$

$a_1,\cdots, a_n \in \phi^{-1}(D), \omega \in \Omega_N$

$\omega(a_1,\cdots,\a_t ...)$ 
 



\end{document}