\documentclass[12pt]{article}

\usepackage{amsmath}
\usepackage{enumerate}

\begin{document}



\section{loeng I}

\subsection{Meenutusi Algebra I-st}

$A \neq \emptyset$
$A^n = A x A x ... A = \{(a_1,\cdots,a_n)| a_i \in A \}$

n - tuple
$|A^0| = 1 (A^0 = \{ \emptyset \}$
$\omega: A^n \rightarrow A$ 
n-kohaline algebraline tehe hulgal A

n-aarne:
\begin{enumerate}
\item n=1 unaarne tehe
\item n=2 binaarne tehe
\item n=0 nullarne tehe
\end{enumerate}


\subsection{$\Omega$-algebra}

Def. 1.1.1 Hulka $\Omega$ nimetakse t\"u\"ubiks ehk signatuuriks kui ta on esitatud mittelõikuvate alamhulkade $\Omega_1 yhend \Omega_2...$

Def 1.1.2 Olgu $\Omega tüüp$. Mittet\"uhja hulka A nimetatakse $\Omega$-algebraks, kui iga $a$ korral igale $\omega \in \Omega_n$ vastab n-aarne tehe hulgal A, mida tähistatakse sama s\"mboliga $\omega$. 

Kui tahetakse rõhutada, mis t\"u\"upi algebraga on tegemist, siis tähistatakse $\Omega$ algebrad paarina $(A;\Omega)$

Näited:

\begin{enumerate}
\item R\"uhmoid - hulk \"he binaarse tehtega, st. $\Omega= \Omega_2 = \{*\}$
\item Poolr\"uhm - sama signatuur mis r\"uhmoidil
\item Monoid - \"uhikelemendiga poolr\"uhm, vaatame seda tihti laiemal signatuuriga, $\Omega = \Omega_0 yhend \Omega_2, \Omega_0 = \{1\}$ - \"uhikelemendi fikseerimine, $\Omega_2 = \{*\}$
\item R\"uhm - saab kirjeldada eelnevate signatuuride kaudu, aga parem kirjeldada järgnevalt: $\Omega = \Omega_0 yhend \Omega_1 yhend \Omega_2$, kus $\Omega_1 = \{ ^{-1}\}$
\item Ring - algebralinie struktuur signatuuriga(signatuuris): $\Omega = \Omega_0 yhend \Omega_1 yhend \Omega_2$, kus $\Omega_2 = \{+,*\}, \Omega_1 = {-(vastandelemendi võtmine}, \Omega_0 = {0,1}$
\item Vektorruum - struktuur signatuuriga:
$\Omega = \Omega_0 yhend \Omega_1 yhend \Omega_2$, kus
$\Omega_2 = \{+\}, \Omega_1 = \{-(vastandelement, pole vajalik kui skalaariga korrutamine sissetoodud)\} yhend {\alpha * | \alpha \in K}, \Omega_0 = \{0\} - samuti avaldatav skalaariga korrutimase kaudu$ 
\end{enumerate}

\subsection{Morfismid}

Def 1.2.1 Kujutust $\phi$ $\Omega$-algebrast $A$ $\Omega$-algebrasse $B$ nimetatakse homomorfismideks, kui iga $n, \omega \in \Omega_n$ ja suvaliste $a_1,...,a_n \in A$ korral kehtib võrdus $\phi(\omega(a_1,...,a_n)) = \omega(\phi(a_1),\cdots, \phi(a_n))$

$Hom(A,B) - \{\phi | \phi on homoformism A-st B-sse\}$

Näide:

Olgu A, B sellised $\Omega = \{1\}, \{ ^{-1}\}, \{*\}$
$\phi : A \rightarrow B homoformism
\phi(1) = 1, \phi(1) = \phi(1*1) = \phi(1)*\phi(1) \implies \phi(1) = 1 (kolmanda põhjal)
\phi(x^{-1}) = \phi(x)^{-1}, 1 = \phi(1) = \phi(x^{-1}x) = \phi(x^{-1})\phi(x) (kolmanda põhjal)
\phi(xy) = \phi(x)\phi(y)$

Taandub kolmanda omanduse kontrollimisele.


Lineaarkujutis on vektorruumide isomorfism.

Olgu meil $\Omega$-algebrad A,B,C ning nende homoformismid $\phi : A \rightarrow B$, $\psi : B \rightarrow C, (\psi \phi ) = \psi (\phi ( x)), x \in A$. Siis see kompositsioon on samuti homoformism (kui teda saab nii defineerida). 

Tõestame : Veendume, et $(\psi\phi)(\omega(a_1,\cdots, a_n)) = \omega((\psi \phi)(a_1,\cdots,a_n))$. See on samaväärne sellega, te $\psi(\phi(\omega(a_1,\cdots,a_n))) = \psi(w(\phi(a_1),\cdots, \phi(a_n)) ...$

Endomorfism (End(A) = Hom(A,A)).

Lause 1.2.2 (End(A);*) on monoid. 

Tõestus: Assotatiivsus on selge, tuleneb homoformismide omadustest. \"Uhikelement ? $id_A : A \rightarrow B, id_{A}(x) = x, x \in A$

Def 1.2.3 Bijektiivne homomorfismi nimetatakse isomorfismiks. 

Lause 1.2.3 Isomorfism on ekvivalentsiseos kõigi $\Omega$-algebrade klassis, ehk ta on reflektsiivne, s\"ummeetriline ja transitiivne. 

Tõestus:
\begin{enumerate}
\item Refleksiivsus, st. A $isom$ A, $id_A: A \rightarrow A$
\item S\"ummeetria. Olgu $ \phi : A \rightarrow B$ isomorfism. Vaja $ \psi : B \rightarrow A$ mis oleks isomorfism. Valime selleks $\phi^{-1}$ Vaja näidata, et iga  $b_1,\cdots,b_n \in B$ korral $\phi^{-1} (\omega(b_1,\cdots, b_n)) = \omega(\phi^{-1}(b_1), \cdots, \phi^{-1}(b_2))$ Rakendame mõlemale poole $\phi$. $\phi(\omega(b_1,\cdots, b_n)) = \phi \omega(\phi^{-1}(b_1), \cdots, \phi^{-1}(b_2)) .... $
\item Transitiivsus - ise! 
\end{enumerate}

Isomorfismi tähtsus. Kui meid huvitab tehe ja tema omadused, siis need jäävad samaks isomorfismi klassi täpsusega. 

$Aut(A)$ 

Lause 1.2.4 Aut(A) on rühm. 

Tõestus: 
$\phi,\psi \in AutA, $

$ \psi \phi \in End(A), $
 
$ \phi \psi \in AutA $
 
$ id_a \in AutA $
 
$ \phi in AutA \implies \phi^{-1} \in AutA $
 
 Näited:
 
 C kompleksarue korpus
 $\phi: C \rightarrow C, \phi(\alpha) = \overline{\alpha}$
 
 G suvaline r\"uhm
 
 $g \in G$
 
 $\phi : G \rightarrow ...$
 
 $\phi(x) = g^{-1} x g $
 
 \subsection{Alamalgebra}

 def 1.3.1. Mõte : $B \subset A, b_1,\cdots,b_n : \omega^B (b_1,\cdots,b_n) = \omega^A (b_1,\cdots,b_n) ( \in B).$ 
 Algebra alamhulk, mis on kinnine tehete suhtes on alamalgebra.
 
Näide: $(A;*)$ poolr\"uhm. $B \subset A, x,y \in B \implies xy \in B $ , kui $B = \emptyset$, siis ei ole alamalgebra aga rahuldab definitsiooni. Täiendame: 
  Algebra \textbf{mittet\"hi} alamhulk, mis on kinnine tehete suhtes on alamalgebra.
  
 $B \leq A \iff B on A alamalgebra$
 
$ B \leq A, \tau : B \rightarrow A$

$\tau(x) = x, x \in B$

$\tau \in Hom(B,A)$, $\tau$ \"uks\"uhene. 


1.3.1 tõestus:
 
$\phi \in Hom(B,A)$
$\phi $ \"uks\"uhene.
$\phi(B) \subset A$
$B \rightarrow \phi(B) = \{ \phi(x) | x \in B \}$

$phi \in Hom(A,B)$
$C \leq A , D \leq B $

$\phi(C) \leq B , \phi^{-1}(B) \leq A$

Esimese ise. Teine:

$\phi^{-1}(B) \leq A$

$a_1,\cdots, a_n \in \phi^{-1}(D), \omega \in \Omega_N$

$\omega(a_1,\cdots,a_t ...)$ 
 
 
\section{loeng II}

\paragraph{Lause 1.3.3} Olgu antud $\omega$-algebra A alamalgebrate s\"steem $B_i$, $i \in I$, kujsuures $B= yhisosa_{i \in I} B_i \neq \emptyset$ Siis $B \leq A$.

\subparagraph*{Tõestus}

...

Vaatleme alamhulka $X$:
$\emptyset \neq X \subset A$
Vaatleme hulka $yhisosa \{ B | X \leq B \leq A \} \neq \emptyset$. Vastavalt lausele 1.3.3 on tegemist alamalgebraga. Sellist alamalgebrad tähistatakse $<X>$
Kui $<X> = A$ ehk $X$ on $A$ moodustajate s\"usteem.

\subsection{Faktoralgebra}

Eesmärgiks on t\"ukeldada $\omega$-algebra mittelõikuvateks osadeks, nii et nende osade hulgal saaks loomulikul viisil defineerida $\omega$-algebra struktuuri. 

$\rho \in Eqv(A), \rho \subset A x A $, vastab kolmele tingimusele:
\begin{enumerate}
\item refleksiivne
\item transitiivne
\item s\"ummeetriline
\end{enumerate}
$a \in A$, $\{x \in A | a \rho x \} = a \/ \rho$, $a \in a \/ \rho$ Faktorhulgaks $A \/ \rho = \{ a \/ \rho | a \in A \}$ 

$a_1 \/ \rho = a_2 \/ \rho \iff a_1 \rho a_2$

Võtame $\omega \in \Omega_n$, $a_1 / \rho , \cdots , a_n / \rho \in A / \rho$ .

$\omega(a_1 / \rho , \cdots , a_n / \rho) = \omega(a_1,\cdots,a_n)/ \rho$

Lisame $\omega$-le lisatingimiuse : $(x_1,y_i),\cdots, (x_n,y_n)  \in \rho \iff ( \omega(x_1, \cdots , x_n), \omega(y_1,\cdots,y_n)) \in \rho$

Olgu $\rho \in Eqv(a)$.
Eksisteerib kujutis $ \pi : A \rightarrow A/ \rho$, $\pi (a) = a / \rho $ - loomulik kujutus faktorhulgale, projektsioon.


Võtame $\omega \in \Omega_n, a_1,\cdots,a_n \in A$

$\pi(\omega(a1,\cdots,a_n)) = \omega(a_1,\cdots,a_n)/\rho = \omega( a_1 / \rho, \cdots , a_n / \rho) = ...$

\subsection{Def - tuum}


\subsection{Lause 1.4.3}

\subsubsection*{Tõestus}
Olgu $\phi : A \rightarrow B$ homoformism. $\rho$ - $\phi $ tuum . 
Valime $\omega \in \Omega_n$, $a_1,\cdots,a_n,a_1^{`},\cdots,a_n^{`}$. Kas $\omega(a_1,\cdots ,a_n) \rho \omega(a_1^{`},\cdots ,a_n^{`})$ kehtib ? ...

\paragraph{Homomorfismiteoreem}

\paragraph*{Tõestus}
Olgu $\psi : A/ \rho \rightarrow B, \psi(a / \rho) := \phi(a)$. Kas on \"uheselt määratud ? Ehk kas $a_1/ \rho = a_2 / \rho \iff \phi(a_1) = \phi (a_2)$. Siit saaksime kätte ka injektiivsuse. Piisab arvesse võtta, et eelnev tähendab, et $a_1 \rho a_2$, nin kun $\rho$ on $\phi$ tuum. siis on tulemus selge. Sürjektiuuvses tuleb sellest, et $\phi$ sürjektiivne. 
Kas $\psi$ on homoformism ?
Olgu $\omega \in \Omega_n, a_1, \cdots, a_n \in A$.
Siis  $\psi(\omega(a_1 / \rho, \cdots, a_n / \rho) = \psi (\omega(a_1, \cdots, a_n) / \rho) = \phi(w(a_1,\cdots,a_n)) = \omega(\phi(a_1),\cdots, \phi(a_n)) = \omega(\psi(a_1 / \rho, \cdots, a_s / \rho)).$

\paragraph{Lause 1.4.4} Olgu $\rho$ $\Omega $-algebra A kongruents, $D \leq A / \rho$ ning $\pi $ kongurgentsi $\rho $ tuum. Siis $D isomeetriline C / \rho|_C , kus C = \pi^{-1}(D) $.

\paragraph*{Tõestus}

Olgu $\pi^{-1}(D) = C \leq A$. Olgu $\alpha$ $\pi$ ahend C-le ($\alpha = \pi|_{C}$). Siis $\alpha : C \rightarrow D$, $\alpha$ on homomorfism. Väidame, et $\alpha$ on s\"urjektiinve. Kuna $\pi$ oli s\"urjektiinve, siis $\forall x \in A \pi(x) = y$. Seega $\alpha(x) = y$. 

K\"usimus : kui kaks korda faktoriseerime, mis siis juhtub, kas me saame midagi uut ? Võimalik asendada isomorfismi täpsuseni üks kord faktoriseerimisega. 

Olgu antud $\rho$ ja $\sigma$ $\Omega$-algebra A kongurentsid, kusjuures $\rho \leq \sigma, (x,y) \in \rho \implies (x,y) \in \sigma$. Defineerime faktoralgebral $A/ \rho$ binaarse seos:

$\sigma / \rho = \{(x / \rho, y / \rho  | (x,y) \in a \sigma \}$

Võime veenduda, et nii defineertus seos $\sigma / \rho$ on faktoralgebra $A / \rho$ kongurents.

\paragraph{Teoreem 1.4.2}

Olgu $\rho \in Con(A), \tau \in Con(A / \rho)$. $\pi : A \rightarrow A / \rho$. Olgu $x,y \in A$.  Defineerime $\sigma$: $(x,y) \in \sigma \iff \pi(x) \tau \pi(x)$

Väide: $\sigma \in Con(A)$. Veendume, et $\sigma \in Eqv(A)$. Olgu $x,y,z \in A, (x,y), (y,z) \in \tau$, st. $(\pi(x),\pi(y)), (\pi(y),\pi(z)), (\pi(x),\pi(z)) \in \tau$.





\end{document}