\documentclass[12pt]{article}

\usepackage{amsmath,amssymb}
\usepackage{enumerate}


\begin{document}



\section{loeng I}

\subsection{Meenutusi Algebra I-st}

$A \neq \emptyset$
$A^n = A x A x ... A = \{(a_1,\cdots,a_n)| a_i \in A \}$

n - tuple
$|A^0| = 1 (A^0 = \{ \emptyset \}$
$\omega: A^n \rightarrow A$ 
n-kohaline algebraline tehe hulgal A

n-aarne:
\begin{enumerate}
\item n=1 unaarne tehe
\item n=2 binaarne tehe
\item n=0 nullarne tehe
\end{enumerate}


\subsection{$\Omega$-algebra}

Def. 1.1.1 Hulka $\Omega$ nimetakse t\"u\"ubiks ehk signatuuriks kui ta on esitatud mittelõikuvate alamhulkade $\Omega_1 yhend \Omega_2...$

Def 1.1.2 Olgu $\Omega$ tüüp. Mittet\"uhja hulka A nimetatakse $\Omega$-algebraks, kui iga $a$ korral igale $\omega \in \Omega_n$ vastab n-aarne tehe hulgal A, mida tähistatakse sama s\"mboliga $\omega$. 

Kui tahetakse rõhutada, mis t\"u\"upi algebraga on tegemist, siis tähistatakse $\Omega$ algebrad paarina $(A;\Omega)$

Näited:

\begin{enumerate}
\item R\"uhmoid - hulk \"he binaarse tehtega, st. $\Omega= \Omega_2 = \{*\}$
\item Poolr\"uhm - sama signatuur mis r\"uhmoidil
\item Monoid - \"uhikelemendiga poolr\"uhm, vaatame seda tihti laiemal signatuuriga, $\Omega = \Omega_0 yhend \Omega_2, \Omega_0 = \{1\}$ - \"uhikelemendi fikseerimine, $\Omega_2 = \{*\}$
\item R\"uhm - saab kirjeldada eelnevate signatuuride kaudu, aga parem kirjeldada järgnevalt: $\Omega = \Omega_0 yhend \Omega_1 yhend \Omega_2$, kus $\Omega_1 = \{ ^{-1}\}$
\item Ring - algebralinie struktuur signatuuriga(signatuuris): $\Omega = \Omega_0 yhend \Omega_1 yhend \Omega_2$, kus $\Omega_2 = \{+,*\}, \Omega_1 = {-(vastandelemendi võtmine}, \Omega_0 = {0,1}$
\item Vektorruum - struktuur signatuuriga:
$\Omega = \Omega_0 yhend \Omega_1 yhend \Omega_2$, kus
$\Omega_2 = \{+\}, \Omega_1 = \{-(vastandelement, pole vajalik kui skalaariga korrutamine sissetoodud)\} yhend {\alpha * | \alpha \in K}, \Omega_0 = \{0\} - samuti avaldatav skalaariga korrutimase kaudu$ 
\end{enumerate}

\subsection{Morfismid}

Def 1.2.1 Kujutust $\phi$ $\Omega$-algebrast $A$ $\Omega$-algebrasse $B$ nimetatakse homomorfismideks, kui iga $n, \omega \in \Omega_n$ ja suvaliste $a_1,...,a_n \in A$ korral kehtib võrdus $\phi(\omega(a_1,...,a_n)) = \omega(\phi(a_1),\cdots, \phi(a_n))$

$Hom(A,B) - \{\phi | \phi on homoformism A-st B-sse\}$

Näide:

Olgu A, B sellised $\Omega = \{1\}, \{ ^{-1}\}, \{*\}$
$\phi : A \rightarrow B homoformism
\phi(1) = 1, \phi(1) = \phi(1*1) = \phi(1)*\phi(1) \implies \phi(1) = 1 (kolmanda põhjal)
\phi(x^{-1}) = \phi(x)^{-1}, 1 = \phi(1) = \phi(x^{-1}x) = \phi(x^{-1})\phi(x) (kolmanda põhjal)
\phi(xy) = \phi(x)\phi(y)$

Taandub kolmanda omanduse kontrollimisele.


Lineaarkujutis on vektorruumide isomorfism.

Olgu meil $\Omega$-algebrad A,B,C ning nende homoformismid $\phi : A \rightarrow B$, $\psi : B \rightarrow C, (\psi \phi ) = \psi (\phi ( x)), x \in A$. Siis see kompositsioon on samuti homoformism (kui teda saab nii defineerida). 

Tõestame : Veendume, et $(\psi\phi)(\omega(a_1,\cdots, a_n)) = \omega((\psi \phi)(a_1,\cdots,a_n))$. See on samaväärne sellega, te $\psi(\phi(\omega(a_1,\cdots,a_n))) = \psi(w(\phi(a_1),\cdots, \phi(a_n)) ...$

Endomorfism (End(A) = Hom(A,A)).

Lause 1.2.2 (End(A);*) on monoid. 

Tõestus: Assotatiivsus on selge, tuleneb homoformismide omadustest. \"Uhikelement ? $id_A : A \rightarrow B, id_{A}(x) = x, x \in A$

Def 1.2.3 Bijektiivne homomorfismi nimetatakse isomorfismiks. 

Lause 1.2.3 Isomorfism on ekvivalentsiseos kõigi $\Omega$-algebrade klassis, ehk ta on reflektsiivne, s\"ummeetriline ja transitiivne. 

Tõestus:
\begin{enumerate}
\item Refleksiivsus, st. A $isom$ A, $id_A: A \rightarrow A$
\item S\"ummeetria. Olgu $ \phi : A \rightarrow B$ isomorfism. Vaja $ \psi : B \rightarrow A$ mis oleks isomorfism. Valime selleks $\phi^{-1}$ Vaja näidata, et iga  $b_1,\cdots,b_n \in B$ korral $\phi^{-1} (\omega(b_1,\cdots, b_n)) = \omega(\phi^{-1}(b_1), \cdots, \phi^{-1}(b_2))$ Rakendame mõlemale poole $\phi$. $\phi(\omega(b_1,\cdots, b_n)) = \phi \omega(\phi^{-1}(b_1), \cdots, \phi^{-1}(b_2)) .... $
\item Transitiivsus - ise! 
\end{enumerate}

Isomorfismi tähtsus. Kui meid huvitab tehe ja tema omadused, siis need jäävad samaks isomorfismi klassi täpsusega. 

$Aut(A)$ 

Lause 1.2.4 Aut(A) on rühm. 

Tõestus: 
$\phi,\psi \in AutA, $

$ \psi \phi \in End(A), $
 
$ \phi \psi \in AutA $
 
$ id_a \in AutA $
 
$ \phi in AutA \implies \phi^{-1} \in AutA $
 
 Näited:
 
 C kompleksarue korpus
 $\phi: C \rightarrow C, \phi(\alpha) = \overline{\alpha}$
 
 G suvaline r\"uhm
 
 $g \in G$
 
 $\phi : G \rightarrow ...$
 
 $\phi(x) = g^{-1} x g $
 
 \subsection{Alamalgebra}

 def 1.3.1. Mõte : $B \subset A, b_1,\cdots,b_n : \omega^B (b_1,\cdots,b_n) = \omega^A (b_1,\cdots,b_n) ( \in B).$ 
 Algebra alamhulk, mis on kinnine tehete suhtes on alamalgebra.
 
Näide: $(A;*)$ poolr\"uhm. $B \subset A, x,y \in B \implies xy \in B $ , kui $B = \emptyset$, siis ei ole alamalgebra aga rahuldab definitsiooni. Täiendame: 
  Algebra \textbf{mittet\"hi} alamhulk, mis on kinnine tehete suhtes on alamalgebra.
  
 $B \leq A \iff B on A alamalgebra$
 
$ B \leq A, \tau : B \rightarrow A$

$\tau(x) = x, x \in B$

$\tau \in Hom(B,A)$, $\tau$ \"uks\"uhene. 


1.3.1 tõestus:
 
$\phi \in Hom(B,A)$
$\phi $ \"uks\"uhene.
$\phi(B) \subset A$
$B \rightarrow \phi(B) = \{ \phi(x) | x \in B \}$

$phi \in Hom(A,B)$
$C \leq A , D \leq B $

$\phi(C) \leq B , \phi^{-1}(B) \leq A$

Esimese ise. Teine:

$\phi^{-1}(B) \leq A$

$a_1,\cdots, a_n \in \phi^{-1}(D), \omega \in \Omega_N$

$\omega(a_1,\cdots,a_t ...)$ 
 
 
\section{loeng II}

\paragraph{Lause 1.3.3} Olgu antud $\omega$-algebra A alamalgebrate s\"steem $B_i$, $i \in I$, kujsuures $B= yhisosa_{i \in I} B_i \neq \emptyset$ Siis $B \leq A$.

\subparagraph*{Tõestus}

...

Vaatleme alamhulka $X$:
$\emptyset \neq X \subset A$
Vaatleme hulka $yhisosa \{ B | X \leq B \leq A \} \neq \emptyset$. Vastavalt lausele 1.3.3 on tegemist alamalgebraga. Sellist alamalgebrad tähistatakse $<X>$
Kui $<X> = A$ ehk $X$ on $A$ moodustajate s\"usteem.

\subsection{Faktoralgebra}

Eesmärgiks on t\"ukeldada $\omega$-algebra mittelõikuvateks osadeks, nii et nende osade hulgal saaks loomulikul viisil defineerida $\omega$-algebra struktuuri. 

$\rho \in Eqv(A), \rho \subset A x A $, vastab kolmele tingimusele:
\begin{enumerate}
\item refleksiivne
\item transitiivne
\item s\"ummeetriline
\end{enumerate}
$a \in A$, $\{x \in A | a \rho x \} = a \/ \rho$, $a \in a \/ \rho$ Faktorhulgaks $A \/ \rho = \{ a \/ \rho | a \in A \}$ 

$a_1 \/ \rho = a_2 \/ \rho \iff a_1 \rho a_2$

Võtame $\omega \in \Omega_n$, $a_1 / \rho , \cdots , a_n / \rho \in A / \rho$ .

$\omega(a_1 / \rho , \cdots , a_n / \rho) = \omega(a_1,\cdots,a_n)/ \rho$

Lisame $\omega$-le lisatingimiuse : $(x_1,y_i),\cdots, (x_n,y_n)  \in \rho \iff ( \omega(x_1, \cdots , x_n), \omega(y_1,\cdots,y_n)) \in \rho$

Olgu $\rho \in Eqv(a)$.
Eksisteerib kujutis $ \pi : A \rightarrow A/ \rho$, $\pi (a) = a / \rho $ - loomulik kujutus faktorhulgale, projektsioon.


Võtame $\omega \in \Omega_n, a_1,\cdots,a_n \in A$

$\pi(\omega(a1,\cdots,a_n)) = \omega(a_1,\cdots,a_n)/\rho = \omega( a_1 / \rho, \cdots , a_n / \rho) = ...$

\subsection{Def - tuum}


\subsection{Lause 1.4.3}

\subsubsection*{Tõestus}
Olgu $\phi : A \rightarrow B$ homoformism. $\rho$ - $\phi $ tuum . 
Valime $\omega \in \Omega_n$, $a_1,\cdots,a_n,a_1^{`},\cdots,a_n^{`}$. Kas $\omega(a_1,\cdots ,a_n) \rho \omega(a_1^{`},\cdots ,a_n^{`})$ kehtib ? ...

\paragraph{Homomorfismiteoreem}

\paragraph*{Tõestus}
Olgu $\psi : A/ \rho \rightarrow B, \psi(a / \rho) := \phi(a)$. Kas on \"uheselt määratud ? Ehk kas $a_1/ \rho = a_2 / \rho \iff \phi(a_1) = \phi (a_2)$. Siit saaksime kätte ka injektiivsuse. Piisab arvesse võtta, et eelnev tähendab, et $a_1 \rho a_2$, nin kun $\rho$ on $\phi$ tuum. siis on tulemus selge. Sürjektiuuvses tuleb sellest, et $\phi$ sürjektiivne. 
Kas $\psi$ on homoformism ?
Olgu $\omega \in \Omega_n, a_1, \cdots, a_n \in A$.
Siis  $\psi(\omega(a_1 / \rho, \cdots, a_n / \rho) = \psi (\omega(a_1, \cdots, a_n) / \rho) = \phi(w(a_1,\cdots,a_n)) = \omega(\phi(a_1),\cdots, \phi(a_n)) = \omega(\psi(a_1 / \rho, \cdots, a_s / \rho)).$

\paragraph{Lause 1.4.4} Olgu $\rho$ $\Omega $-algebra A kongruents, $D \leq A / \rho$ ning $\pi $ kongurgentsi $\rho $ tuum. Siis $D isomeetriline C / \rho|_C , kus C = \pi^{-1}(D) $.

\paragraph*{Tõestus}

Olgu $\pi^{-1}(D) = C \leq A$. Olgu $\alpha$ $\pi$ ahend C-le ($\alpha = \pi|_{C}$). Siis $\alpha : C \rightarrow D$, $\alpha$ on homomorfism. Väidame, et $\alpha$ on s\"urjektiinve. Kuna $\pi$ oli s\"urjektiinve, siis $\forall x \in A \pi(x) = y$. Seega $\alpha(x) = y$. 

K\"usimus : kui kaks korda faktoriseerime, mis siis juhtub, kas me saame midagi uut ? Võimalik asendada isomorfismi täpsuseni üks kord faktoriseerimisega. 

Olgu antud $\rho$ ja $\sigma$ $\Omega$-algebra A kongurentsid, kusjuures $\rho \leq \sigma, (x,y) \in \rho \implies (x,y) \in \sigma$. Defineerime faktoralgebral $A/ \rho$ binaarse seos:

$\sigma / \rho = \{(x / \rho, y / \rho  | (x,y) \in a \sigma \}$

Võime veenduda, et nii defineertus seos $\sigma / \rho$ on faktoralgebra $A / \rho$ kongurents.

\paragraph{Teoreem 1.4.2}

Olgu $\rho \in Con(A), \tau \in Con(A / \rho)$. $\pi : A \rightarrow A / \rho$. Olgu $x,y \in A$.  Defineerime $\sigma$: $(x,y) \in \sigma \iff \pi(x) \tau \pi(x)$

Väide: $\sigma \in Con(A)$. Veendume, et $\sigma \in Eqv(A)$. Olgu $x,y,z \in A, (x,y), (y,z) \in \tau$, st. $(\pi(x),\pi(y)), (\pi(y),\pi(z)), (\pi(x),\pi(z)) \in \tau$.


\section{loeng IV}

\subsection{Lagrange'i teoreem}

Lõpliku rühma järk( elementide arv) jagub tema iga alamhulga järguga.

\subsection{$\Omega$-algebrate otsekorrutis}

Viis kuidas saada mitmest algebrast uus algebra.

Võime defineerida funktsioonid, mis kirjeldavad jadasid. $ \phi : \mathbb{N} \rightarrow \cup_{i \in \mathbb{N}} A_i$, mis rahuldab tingimust $ \phi (i) \in A_i$, iga $i \in \mathbb{N}$ korral.

Projektsioonid - seavad jadale vastavuse mingi kindla elemendi. Tähistame $\pi _ i$. 

\paragraph{ 1.6.1}

\subparagraph{Tõestus}
$\omega \in \Omega_n, a^1 = (a_i^1)_{i \in I},..., a^n = (a_i^n)_{i \in I} ...$ 

\subsection{Võred}

\paragraph{(Osaliselt) Järjestatud hulk} 

Binaarne seas, mis on reflektsiivne, transitiivne ja antis\"meetriline. Lineaarselt järjestatud hulk on selline, kus iga element on mingis seoses iga teisega.

\paragraph{Teoreem 2.2.1}

\subparagraph{Tõestus}

4) Neeldevus (absorbtion)

Tarvilikkus:

$x \leq y \iff x = x alumineraja y$

\section{Loeng V}

$[a,b] = \{x \in L | a \leq x \leq b \}$

$Con(A/\rho) \leftarrow \rightarrow \{\sigma \in Con(A) | \rho \leq \sigma \}$

\paragraph{Teoreem 2.2.2}

Distributiivsed võred. 

\paragraph{Lause 2.3.1}

Ahelad on distributiivsed võred.

\paragraph{Lause 2.3.2}

Tähtis distributiivne võre $(P(A); \
intersection; union)$

Isendega duaalsus.

\paragraph{Lause 2.3.3}

\paragraph{Järeldus 2.3.1}

\paragraph{Teoreem 2.3.1}

Võre on modulaarne parajasti siis, kui ta ei oma võrega $N_5$ isomorfset alamvõret. Modulaarne võre on distributiivne parajasti siis, kui ta ei oma võrega $M_3$ isomorfset alamvõret.

\subparagraph{Tõestus}

Riina esitab seminaris.

\paragraph{Teorem 2.4.1}
Võre on distributiivne parajasti siis, kui ta on isomorfne mingi hulga kõigi alamhulkade võre mingi alamvõrega. 

\paragraph{Definitsioon 2.4.1}
Võre mittetühja alamhulka F nimetatakse filtriks, kui ta on kinnine alumise raja võtmise suhtes ja koos iga elemendiga $a$ sisaldab ka võre $L$ kõik elemendist $a$ suuremad elemendid.

\subparagraph{Märkus}
Filtri ja algfiltri duaalsed mõisted on vastavalt ideaal ja algideal.

\paragraph{Definitsioon 2.4.2}
Võre $L$ filtrit $F$ nimetatakse algfiltriks, kui sellest, et $a V b \in F$, kus $a,b \in L$, järjeldub
$a \in F$ või $b \in F$. Algfilter $F \neq L$. 

\paragraph{Zorni lemma}
Olgu meil järjestatud hulk $A$. Eeldame, et iga hulga $A$ alamhulk omab ülemist tõket hulgas $A$. Siis sellest järeldub, et $A$ omab vähemalt ühte maksimaalset elementi. $C \subset A alamhulk: x,y \in C \implies x \leq y või y \leq x$.

\paragraph{Lause 2.4.1}
Distributiivse võre iga kahe erivena elemendi jaoks leidub algfilter, mis sisaldab täpselt ühte neist kahest. 

\subparagraph{Tõestus}


\paragraph{Teoreem 2.4.1}
Võre on distributiivne parajasti siis, kui ta on isomorfne mingi hulga kõigi alamhulkade võre mingi alamvõrega

\subparagraph{Selgitus}
Olgu L distributiivne võre. Vaja ledia hulk A ja \"uks\"uhene homomorfism $\Phi : L \rightarrow P(A), \Phi(L) \leq P(A), L isomm \Phi(L)$. 

\subparagraph{Tõestus}

\section{R\"uhmad}

\subsection{Faktorr\"uhma faktoriseerimine}

\paragraph{Isomorfismiteoreem}
Olgu $H$ rühma $G$ normaalne alamr\"uhm, B r\"uhma G alamr\"uhm ning A r\"uhma B normaalne alamr\"uhm. Siis $BH/AH isom B/(A(B yhisosa H))$.

\paragraph{Järeldus 3.2.1.}
Olgu $H$ r\"uhma G normaalne alamr\"uhm ja $A$ r\"uhma G alamr\"uhm. Siis $BH/H isom B/(B yhisosa H)$.

\paragraph{Teoreem 3.2.2. (Zassenhausi lemma)}
Kui $H,H',K$ ja $K'$ on rühma G alamrühmad, kusjuures $H'$ on normaalne alamr\"uhm r\"uhmas $H$ ja $K'$ on normaalne alamr\"uhm r\"uhmas K, siis

$(H yhis K)H'/(H yhis K')H' isom (K yhis H)K'/(K yhis H')K'$.

\subparagraph{Tõestus}

Idee: näitame, et mõlemad on isomorfsed $H yhis K / (H' yhis K)(H yhis K')$.
$H'(H yhis K) / H'(H yhis K') isom H yhis K / (H' yhis K)(H yhis K')$.
Kasutame isomorfismiteoreemi. Võtame $B$ rolli $H yhis K$, $H$ rolli sobib  $H'$, $A$ rolli võtame $H yhis K'$. Lisaks vaatama $G$ rollis $H$-d. Kas $h yhis K' normaalne alamryhm H yhis K$?

\subsection{Normaal- ja kompositsioonijadad}

\paragraph{Schreieri teoreem} Antud r\"uhmas suvalised kaks normaaljada omavad ekvivalentseid tihedusi.

\subparagraph{Tõestus}
$\{1\} = H_0 <d H_1 <d H_2 ... H_m= G$

$\{1\}] K_0 <d K_1 <d K_2 ... <d H_n = G$

Defineerime $H_{ij} = H_i(H_{i+1} yhisosa K_j)$ ja $K_{ji} = K_j (K_{j+1} yhisosa H_i)$. 


Miks $H_{ij} <d H_{i,j+1}$ ?


Miks $H_i( H_{i+1} yhisosa K_j) <d H_i(H_{i+1} yhisosa K_{j+1})$ ?

\paragraph{Näide}

Olgu $ m=2, n =3$. Siis peavad eelneva põhjal ekvivalentsed olema $H_0 = H_{00} \leq H_{01} \leq H_{02} \leq H_{03} = H_1=H_{10} \leq H_{11} \leq H_{12} \leq H_{13} = H_2 = G$ ja $K_0 = K_{00} \leq K_{01} \leq K_{02} = K_1 = K_{10} \leq K_{11} \leq K_{12} = K_{2} = K_{20} \leq K_{21} \leq K_{22} = K_3 = G$.

Veenduda Sachenhausi lemma põhjal. 

$H_{01} / H_{00} isomorfne K_{01} / K_{00}$

$H_{02}/ H_{01} isomorfne K_{11}/K_{10}$

$H_{03}/ H_{02} isomorfne K_{21}/K_{20}$

$H_{11} / H_{10} isomorfne K_{02}/K_{01}$

$H_{12} / H_{11} isomorfne K_{12}/K_{11}$

$H_{13} / H_{12} isomorfe K_{22}/K_{12}$

\section{Lihtsad r\"uhmad}

\paragraph{Lause 3.4.1 Abeli rühm on lihtne siis ja ainult siis, kui tema järk on algarv}

\subparagraph{Tõestus}
Kuna alamrühma järk jagab rühma järke, siis algarvulise järguga rühmal saab olla ainult 2 alamrühma - kogu rühm ja 1 elemendiline rühm.

Teistpidi, olgu $A$ lihtne Abeli rühm. $(A,+)$,$0 \neq a \in A$, $ \{na | n \in \mathbb{Z}\}$. Kusjuures, kui $n >0 $ siis $na = a + a + .... + a$, kui $n=0$ siis $0a = 0$. Ja kui $n < 0$ siis $(-n)*a = -(na)$. Elemendi A poolt tektitatdu ts\"ukliline alamr\"uhm. Abeli rühma alamr\"uhm on lihte, seega $A = < a >$. 

\paragraph{Teoreem 3.4.1 Kui $n = 3$ või $n \geq 5$, siis rühm $A_n$ on lihtne }
   
\paragraph{Teoreem 3.4.2} Kui $n > 2$ või $n=2$ ja $|K| > 3$, siis projektiivne spetsiaalne lineaarr\"uhm PSL(n,K) on lihtne. 

\section{Lahenduvad rühmad}

\paragraph{Definitsioon 3.5.1.} R\"uhma, mis omab normaaljada, mille kõik faktorid on Abeli rühmad, nimetatakse lahenduvaks.

\paragraph{Teoreem 3.5.1} Lahenduva r\"uhma alamrühmad ja faktorrühmad on lahenduvad.

\subparagraph{Tõestus} Olgu meil lahenduv rühm $G$. Kehtigu $\{1\} = H_0 <d H_1 <d H_2 <d ... <d H_m = G$. $H_{i+1}/H_i $ on Abeli rühm $i=0,...,n-1$. 

$A \leq G $, $ A_i = A yhiosa H_i$, $A_0 = A yhisosa \{1\} = \{1\}$, $A_n = A yhisosa G = A$, $i \leq j \implies A_i \leq A_j$. 

\paragraph{Teoreem X} Iga paaritu arvulise järguga rühm on lahenduv

\subparagraph{Tõestus}
Olgu $|G|$ paaritu. $\{1\} = H_0 <d H_1 <d H_2 <d ... <d H_n = G$. Kõik jada faktorit lihtsad lõplikud rühmad. Alamrühma järk jagab rühma järku $\implies$ alamr\"uhmade järgud on paaritud. 

\section{Faktorringi faktoriseerimine}

\paragraph{Lause 4.1.1} Kõik korpused on lihtsad ringid. Iga lihtne kommutatiivne ring on korpus. 

\subparagraph{Tõestus} $\{0\} \neq I <d K$. $I$ - ideaal. ...

\paragraph{Lause 4.1.2} Täielik maatriksring $Mat_n (K)$ on lihtne iga naturaalarve $n$ ja korpuse $K$ korral. 

%yks loeng puudu

AlI - iga vektorruum omab baasi lõplikul juhul
lõpmatu mõõtmelise baasi lin sõltumatus - kõik lõpblikud alamhulgad sõltumatud.
T.4.4.2
$S = \left\{ X | X \subset V, X on lin. sõltumatu \right\}$
Zorni lemma eeldute kontroll.
$\left\{ X_i | i \in I \right\}$, $X_i \in S$ Otsime suurimat elementi
$X = \sup_{i \in I} X_i$. Kas $X$ kuulub hulka $S$?
...
Zorni lemma eeldus täidetud.
$S$ omab maksimaalset elementi, olgu selleks $Z$. $Z$ on $V$ baas ? 
Valime $v \in V$, kas $v \in L(Z)$. Oletame, et $v \not \in V$, siis $Z \sup \left\{ v \right\}$ on lin sõltumatu, see on aga vastuolu.

$V$ vektorruum \"ule $K$
$ei, i \in I$ -$V$ baas.
$V isom ringpluss \sum \limits_{i \in I} K_i$
$ringpluss \sum \limits_{i \in I} K_i = \left\{ (k_i)_{i \in I} | k_i \in K, | \left\{ j \in L | k_j = 0 \right\} | < \infty \right\}$

defineerime $\phi: V \to ringpluss \sum \limits_{i \in I} K_i$ nii, et $\phi(v) = (l_i)_{i \in I}, l_i = \begin{cases} k_i, \text{kui} i \in \left\{ i_1,i_2,...,i_n \right\} \\
0, \text{kui} i \not \in \left\{ i_1,i_2,...,i_n \right\}
\end{cases}$

\section{Ringide esitused ja moodulid}
D 4.5.1

$A$ Abeli r\"uhm, End$(A)$ on r\"uhm. Liitimine defineeritud kui $\left( \phi + \psi \right) \left( a \right) = \phi \left( a \right) + \psi \left( a \right)$. 

T 4.5.1
$\phi : R \to End(M;+)$, iga $r \in R$ kollab tekib loomulik kujutus $l_2: M \to M, x \to rx$. Sellest võime mõelda kui vasaknihkest.  $\phi \left( r \right ) = l_r$. Veendume, kas definitsioon on korrektne. Esiteks, kas $l_r \in \text{End}(M;+)$ ? $l_2 (x +y ) = r (x + y ) = rx + ry = l_r(x) + l_r(y)$. Veel, $\phi(rs) = \phi(r) * \phi(s)$, $\phi(1) = 1_M$, $l_{r+s} = l_r + l_s$.

D 4.5.2

$\phi: R \to \text{End}(A)$. Oletame, et $\phi$ on \"uks\"uhene, oletame, et $r$ kuulub Ker$(\phi)$, $\phi(r) = 0 = \phi(0) \implies r=0$, seega Ker$(\phi)$  ....

R-moodul M on täpne $\implies$ vastav esitus on täpne. $\phi : R \to \text{End}(M;+)$, $\phi(r) = l_r$, Ker$(\phi) = \left\{ 0 \right\}$, $l_r(x) = 0 \forall x \in M \iff r=0$

T 4.5.2
$R isom \phi(R) \leq \text{End} \left( M;+ \right)$.

Ainult null element anuleerib kõik mooduli elemendid. 

\section{Abeli r\"uhmad}
POLE SLAIDIL! 
Idee: näidata, et mooduli ehitus võib olla keerulisem. 

Tsükliline $R$-moodul 

Def. $R$-moodulit nimetatakse ts\"ukliliseks, kui ta on tekitatud \"uhe elemendi poolt. 

Olgu $M$ ts\"ukliline $R$-moodul, see tähendab $\exists a \in M, M = < a >$. $M = Ra = \left\{ ar | r \in R \right\}$. $RA \subset < a > $. $ra + sa = (r + s) a$, $s(ra) = (sr)a$. $a = 1 * a \in Ra$. 

L. Iga ts\"ukliline $R$-moodul on isomorfne $R$-mooduli $R$ faktormooduliga. 

\begin{proof}
$M = Ra$.
$\phi: R \to Ra$.
$\phi(r) = ra$.
Kontrollida homomorfismi. S\"urjektiivne, homomorfismi teoreemi põhjal $M$ isomorfne $R/ \text{Ker} \left( \phi \right)$.

\end{proof}



L 4.5.1


\end{document}