\documentclass[12pt]{report}

\usepackage[utf8]{inputenc}
\usepackage{amsmath,amssymb}
\usepackage{amsthm}
\usepackage{enumerate}
\usepackage[estonian]{babel}
\usepackage{color}
\usepackage[usenames,dvipsnames]{xcolor}
\usepackage{float}
\usepackage{theoremref}

\setlength\parindent{0pt}

\definecolor{background_example}{HTML}{EDEDED}

\newcounter{def}
\theoremstyle{definition}
\newtheorem{normaalne_alamruhm}[def]{Definitsioon}
\newtheorem{lihtne_alamruhm}[def]{Definitsioon}
\newtheorem{substitutsioon}[def]{Definitsioon}
\newtheorem{paarissubstitutsioon}[def]{Definitsioon}
\newtheorem{transpositsioon}[def]{Definitsioon}
\newtheorem{tsukel}[def]{Definitsioon}

\theoremstyle{plain}
\newcounter{teoreem}
\newtheorem{paaris_subsitutsioonide_hulk}[teoreem]{Teoreem}
\newtheorem{trans_korrutis}[teoreem]{Teoreem}
\newtheorem{trans_korrutis_sub}[teoreem]{Lause}

\begin{document}

Kui $A$ on mittet\"uhi hulk, siis tähistame $S(A)$ abil kõigi bijektiivsete teisenduste ehk substitutsioonide r\"uhma hulgal $A$. Seda nimetame s\"ummeetriliseks r\"uhmaks hulgal $A$. Kui $A=\left\lbrace 1,...,n \right\rbrace$, siis kirjutame $S(A)$ asemel $S_n$. Kõik paarissubsitutsioonid hulgal $\left\lbrace 1,...,n \right\rbrace$ moodustavad r\"uhma $S_n$ normaalse alamr\"uhma, mida tähistame $A_n$. 
\begin{paaris_subsitutsioonide_hulk}
Kui $n=3$ või $n \geq 5$, siis r\"uhm $A_n$ on lihtne.
\end{paaris_subsitutsioonide_hulk}

\begin{normaalne_alamruhm}
R\"uhma $G$ alamr\"uhma $H$ nimetatakse normaalseks, kui iga $a \in G$ ja $h \in H$  korral $a^{-1}ha \in H$ .
\end{normaalne_alamruhm}
\begin{lihtne_alamruhm}
R\"uhma, millel ei ole mittetriviaalseid normaalseid alamr\"uhmi, nimetatakse lihtsaks r\"uhmaks.
\end{lihtne_alamruhm}
\begin{substitutsioon}
Olgu $M \neq \emptyset$ mistahes hulk. Subsitutsiooniks hulgal $M$ nimetatakse hulga $M$ mistahest \"uk\"uhest pealekujutust.
\end{substitutsioon}
\begin{paarissubstitutsioon}
Substitutsiooni nimetatakse paarissubstitutsiooniks, kui mõlemad permutatsioonid tema esituses tabelina on \"uhesuguse paarsusega, see tähendab, inversioonide arv mõlemas esituses tabelina on sama paarsusega. Vastasel juhul nimetatakse subsitutsiooni paarituks substitutsiooniks.
\end{paarissubstitutsioon}
\begin{transpositsioon}
Lõpliku hulga teisendust, mis seisneb selle hulga kahe erineva elemendi \"umbervahetamises, kusjuures \"ulejäänud elemendid jäävad paika nimetatakse transpositsiooniks. 
\end{transpositsioon}
\begin{trans_korrutis}
Iga substitutsioon vähemalt kaheelemendilisel hulgal on esitatav transpositsioonide korrutisena.
\end{trans_korrutis}
\begin{trans_korrutis_sub}
Tegurite arv substitutsiooni esituses transpositsioonide korrutisena on sama paarsusega kui substitutsioon ise.
\end{trans_korrutis_sub}
\begin{tsukel}
Subsitsiooni nimetatakse ts\"ukliks, kui ta paigutab teatud elemendid ts\"ukliliselt \"umber, \"ulejäänud elemendid jätab aga paigale. Ts\"uklit, mis viib elemendi $i_1$ elemendiks $i_2$, elmendi $i_2$ elemendiks $i_3$,...,elmendi $i_k$ elemendiks $i_1$ tähistataikse $\left(i_1 i_2 ... i_k \right)$.
\end{tsukel}
Mistahes substitutsiooni on võimalik esitada niinimetatud sõltumatute ts\"uklite korrutisena - ts\"uklite korrutisena, mille \"uleskirjutises ei ole \"uhiseid elemente.

Paneme tähele, et transpositsioone võime vaadelda kui kahe elemendilise ts\"ukleid.



\end{document}
